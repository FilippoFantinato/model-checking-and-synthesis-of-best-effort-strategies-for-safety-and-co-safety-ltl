%!TEX root = ../../main.tex

\section{Best-Effort semantics}
In this section we introduce the bounded-value and best-effort semantics.

The \textit{bounded-value semantics} is the first semantics presented because it is the basis for the other one.
Such semantics is useful to add a constraint to a value, ensuring that if a formula satisfies a state-sequence, then such value does not reach a constant value chosen a-priori. 
Note that we will not set the domain for the variable, but as long as the domain $\domain$ of the variable is a strict partially ordered set $\tuple{\domain,\prec}$ where $\prec$ is the strict partial order on $\domain$ (i.e. $\prec$ is irreflexive, antisymmetric, and transitive on $\domain$).
We define the bounded-value semantics for both safety and co-safety fragments.
The definitions in safety and co-safety fragments are different because, while for co-safety properties we need to look only all over the minimal prefix, for safety properties we need to look all over the trace which could be infinite.
The minimal prefix of a co-safety formula is a state-sequence prefix in which the formula holds the first time.
All prefixes containing the minimal prefix are sub-state-sequences satisfying the formula, but we do not mind what happens after the first time the formula has held.

\begin{definition}[Bounded-value semantics for co-safety formulas]
Let $\phi$ be a formula belonging to the co-safety fragment of $\ltl$, $\sigma$ be a state-sequence evaluated at position $i$, $\tuple{\domain,\prec}$ be a strict partially order set, $\sigma.v \in \domain$ a variable and $u \in \domain$ an upper bound to $\sigma.v$.
We say that $\phi$ is satisfied by $\sigma$ with upper-bound $u$ to $\sigma.v$ at position $i$, denoted as $\sigma,i \modelsBV{v \prec u} \phi$, if and only if
\begin{flalign*}
& \text{$\exists j \geq i$ such that $\sigma[i,j] \models \phi$, $\sigma[i,j-1] \not\models \phi$ and $\forall k \in [i,j].\; \sigma_k.v \prec u$}
\end{flalign*}
We say $\sigma \modelsBV{v \prec u} \phi$ if and only if $\sigma, 0 \modelsBV{v \prec u} \phi$.
\end{definition}

Differently from co-safety formulas where we can fix a point in time up to evaluate the formula, safety formulas are evaluated on possibly infinite state-sequences.
For this reason we define the bounded-value semantics for safety formulas inductively on the structure of safety formulas.
In order to avoid the formulation of bounded-value semantic for a negated generic formula, we consider only formulas in Negated Normal Form (NNF). 
The temporal operators needed to define safety formulas in NNF are: \textit{next}, \textit{yesterday}, \textit{release}, \textit{trigger} and \textit{since}.
Moreover, it is possible to define all temporal operators which are abbreviations of the ones listed here above that do not require negations in front of themselves, such as \textit{globally}, \textit{historically} and \textit{once}.
The basic idea is to check that the value of the chosen variable is related to the bound at every step.
The semantics is almost the same as the regular one, but we check the condition whenever we meet a proposition (negated or not) and whenever we take a step forward or backward.
In any other cases, the semantics does not change.
In fact, all abbreviations and equivalences holding in other safety dialects of $\ltl$, holds here too.

\begin{definition}[Bounded-value semantics for safety formulas]
Let $\phi$ be a formula belonging to the safety fragment of $\ltl$, $\sigma$ be a state-sequence evaluated at position $i$, $\tuple{\domain,\prec}$ be a strict partially order set, $\sigma.v \in \domain$ a variable and $u \in \domain$ an upper bound to $\sigma.v$.
We say that $\phi$ is satisfied by $\sigma$ with upper-bound $u$ to $\sigma.v$ at position $i$, denoted as $\sigma,i \modelsBV{v \prec u} \phi$, if and only if
\begin{flalign*}
&\sigma,i \modelsBV{v \prec u} p &\iff& & & p \in \sigma_i \land \sigma_i.v \prec u \\
&\sigma,i \modelsBV{v \prec u} \ltlNeg{p} &\iff& & & p \not\in \sigma_i \land \sigma_i.v \prec u \\
&\sigma,i \modelsBV{v \prec u} \ltlOr{\phi_1}{\phi_2} &\iff& & & \text{$\sigma,i \modelsBV{v \prec u} \phi_1$ or $\sigma,i \modelsBV{v \prec u} \phi_2$} \\
&\sigma,i \modelsBV{v \prec u} \ltlAnd{\phi_1}{\phi_2} &\iff& & & \text{$\sigma,i \modelsBV{v \prec u} \phi_1$ and $\sigma,i \modelsBV{v \prec u} \phi_2$} \\
&\sigma,i \modelsBV{v \prec u} \ltlX{\phi} &\iff& & & \sigma,i+1 \modelsBV{v \prec u} \phi \\
&\sigma,i \modelsBV{v \prec u} \ltlY{\phi} &\iff& & & \text{$i>0$ and $\sigma,i-1 \modelsBV{v \prec u} \phi$} \\
&\sigma,i \modelsBV{v \prec u} \ltlR{\phi_1}{\phi_2} &\iff& & & \text{either for all $j \geq i. \; \sigma,j \modelsBV{v \prec u} \phi_2$,} \\
& & & & & \text{or there exists $j \geq i$ such that $\sigma,j \modelsBV{v \prec u} \phi_1$ and} \\
& & & & & \text{$\sigma,w \modelsBV{v \prec u} \phi_2$ for all $i \leq w \leq j$} \\
&\sigma,i \modelsBV{v \prec u} \ltlS{\phi_1}{\phi_2} & \iff & & & \text{there exists $0 \leq j \leq i$ such that $\sigma,j \modelsBV{v \prec u} \phi_2$ and} \\
& & & & & \text{$\sigma,w \modelsBV{v \prec u} \phi_1$ for all $j < w \leq i$} \\
&\sigma,i \modelsBV{v \prec u} \ltlT{\phi_1}{\phi_2} & \iff & & & \text{either for all $0 \leq j \leq i. \; \sigma,j \modelsBV{v \prec u} \phi_2$,} \\
& & & & & \text{or there exists $j \in [0,i]$ such that $\sigma,j \modelsBV{v \prec u} \phi_1$ and} \\
& & & & & \text{$\sigma,w \modelsBV{v \prec u} \phi_2$ for all $j \leq w \leq i$}
\end{flalign*}
We say $\sigma \modelsBV{v \prec u} \phi$ if and only if $\sigma, 0 \modelsBV{v \prec u} \phi$.
\end{definition}

The bounded-value semantics is the basis of the best-effort semantics. 
The \textit{best-effort semantics} imposes to satisfy a formula with bounded-value semantics, where the bound is the least upper bound of values satisfying the formula with bounded-value semantics.
Therefore, there is no other value strictly smaller than the chosen one such that a state-sequence models a formula.
The best-effort semantics is useful because it imposes a constraint also to the upper bound.

\begin{definition}[Best-effort semantics]
Let $\phi$ be a formula belonging to either safety and co-safety fragment of $\ltl$, $\sigma$ be a state-sequence evaluated at position $i$, $\tuple{\domain,\prec}$ be a strict partially order set, $\sigma.v \in \domain$ a variable and $u \in \domain$ an upper bound to $\sigma.v$.
We say that $\phi$ is satisfied by $\sigma$ best-effort $u$ to $\sigma.v$ at position $i$, denoted as $\sigma, i \modelsBE{v \prec u} \phi$, if and only if
\begin{flalign*}
\sigma, i \modelsBV{v \prec u} \phi \land
\forall u' \prec u \; \sigma, i \not\modelsBV{v \prec u'} \phi
\end{flalign*}
We say $\sigma \modelsBV{v \prec u} \phi$ if and only if $\sigma, 0 \modelsBV{v \prec u} \phi$.
\end{definition}

\subsection{Notable examples}
In this sub-section we present two use-cases of bounded-value semantics and best-effort semantics. 
For these examples We consider the strict partially ordered set $\tuple{\Nat,<}$ and the corresponding relation $>$. Note that since the relation $<$ on the set of natural numbers has only one previous element for any number, to check whether a formula is satisfied best-effort with bound $u$, we just need to check whether is not satisfied with bound $u$ but not with bound $u-1$.

The first example regards a co-safety property. Suppose we are driving a car to reach a given position on a grid. In this case the plant is the car, while you, the driver, are the controller.
The car knows its internal position $(x,y)$ and has a battery level. 
At each step the battery decreases by $1$, starting from $100$. 
Whenever the battery level reaches $0$ the car cannot perform any further move.
Available moves are: up, down, left, right, right up, right down, left up and left down.
So we can move in all $8$ directions.

\begin{lstlisting}[language=smv, caption=Best-effort semantics: Car example (1)]
MODULE Car(move)
VAR
    x : 0..width;
    y : 0..length;
    battery : 0..max_battery;
DEFINE
    length := 20;
    width  := 20;
    starting_x := 0;
    starting_y := 0;
    max_battery := 100;
ASSIGN
    init(x) := starting_x;
    init(y) := starting_y;
    init(battery) := max_battery;
ASSIGN
    next(x) := case
        battery!=0 & (move=right | move=right_up | move=right_down) & x<width : x + 1;
        battery!=0 & (move=left | move=left_up | move=left_down) & x>0  : x - 1;
        TRUE : x;
    esac;
    next(y) := case
        battery!=0 & (move=up | move=left_up | move=right_up) & y<length : y + 1;
        battery!=0 & (move=down | move=left_down | move=right_down) & y>0  : y - 1;
        TRUE : y;battery!=0
    esac;
    next(battery) := case 
        move!=stop & battery > 0 : battery - 1;
        TRUE : battery;
    esac;
\end{lstlisting}

Let us define the first controller for the car: Driver1.
Driver1 accepts in input the position $(x,y)$ given by Car and aims to reach the position $(15,7)$ on the grid, but this driver does not know that it is possible to perform diagonal moves, so it goes only up, down, left and right.
The implemented strategy by Driver1 is to reach before the same x-axis as the goal one and the reach the same y-axis, ending up in the goal position.

\begin{lstlisting}[language=smv, caption=Best-effort semantics: Car example (2)]
MODULE Driver1(x,y)
VAR
    move : {up,down,right,left,
            right_up,right_down,
            left_up,left_down,
            stop};
DEFINE 
    goal_x := 15;
    goal_y := 7;
ASSIGN
    move := case
        x < goal_x  : right;
        x > goal_x  : left;
        y < goal_y  : up;
        y > goal_y  : down;
        TRUE : stop;
    esac;
\end{lstlisting}

Putting in closed-loop Car and Driver1, we can verify some properties on this system.
The easiest property is to verify whether Driver1 reaches the goal position, but keeping the value of the battery greater than $76$.
In bounded-value semantics, it would be something like that: 
\begin{flalign*}
    \closedloop{\text{C}}{\text{D}} \modelsBV{-\text{C.battery} < -76} \ltlF{(\ltlAnd{\text{C.x} = \text{D.goal\_x}}{\text{C.y} = \text{D.goal\_y}})}
\end{flalign*}

Note that since the bounded-value semantics is defined with a $<$ constraint, but we want to describe a $>$ constraint, it is enough multiply both terms by $-1$ in order to invert the inequality sign.
By running any the model checker NuSMV, we see that Driver1 reaches the goal position by always keeping the battery level greater than $76$.
We note that by lowering the bound $-76$ by $1$, so obtaining $-77$, the formula is no more satisfied, therefore we can say that such closed-loop satisfies best-effort the formula.

Now the question becomes, is Driver1 the optimal controller for Car?
To answer this question we need to show either another controller which satisfies the formula by keeping the battery level less than or equal to $-77$, or the not existence of such a controller.
In this case it is easy to see that a better controller is one which moves diagonally.
Driver2 is an example of controller moving also diagonally, which reaches the goal position by keeping the battery level $> 77$.
Therefore, Driver1 was not the optimal controller for that plant and that formula.

\begin{lstlisting}[language=smv, caption=Best-effort semantics: Car example (3)]
MODULE Driver2(x,y)
VAR
    move : {up,down,right,left,
            right_up,right_down,
            left_up,left_down,
            stop};
DEFINE 
    goal_x := 15;
    goal_y := 7;
ASSIGN
    move := case
        x < goal_x & y < goal_y  : right_up;
        x < goal_x & y > goal_y  : right_down;
        x < goal_x & y = goal_y  : right;
        x > goal_x & y < goal_y  : left_up;
        x > goal_x & y > goal_y  : left_down;
        x > goal_x & y = goal_y  : left;
        y < goal_y  & x = goal_x : up;
        y > goal_y  & x = goal_x : down;
        TRUE : stop;
    esac;
\end{lstlisting}

Regarding safety properties, we consider a water boiler with a controller which must keep the water temperature between $30$ and $70$ degrees.
The boiler, when it is on increments the temperature by $1$, while when it is off the water temperature drops one degree at a time.
We also consider the energy consumption to warm the water, which is exactly the number of steps in which the boiler is on.
Whenever the controller turns off the boiler, the energy consumption is set to $0$.

\begin{lstlisting}[language=smv, caption=Best-effort semantics: Boiler example (1)]
MODULE Boiler(on)
VAR 
    temp  : 0..100;
    energy : 0..100;
ASSIGN
    init(temp) := 50;
    next(temp) := case
        temp=100 : 100;
        temp=0 : 0;
        on  : temp + 1;
        !on : temp - 1;
    esac;
    init(energy) := 0;
    next(energy) := case
        !on : 0;
        on & energy < 100 : energy + 1;
        TRUE : energy;
    esac;
\end{lstlisting}

We define Controller1 in such a way that turns on the boiler exactly when the water temperature goes lower than $40$ degrees and turns off the boiler when the water temperature reaches $50$ degrees.

\begin{lstlisting}[language=smv, caption=Best-effort semantics: Boiler example (2)]
MODULE Controller1(temp)
VAR
    on : boolean;
ASSIGN
    init(on) := FALSE;
    next(on) := case
        temp <= 40 : TRUE;
        temp >= 50 : FALSE;
        TRUE : on;
    esac;
\end{lstlisting}

We can verify some properties on the closed-loop, like verifying that the water temperature is always between the required range and we use little energy as possible.

\begin{flalign*}
    \closedloop{\text{B}}{\text{C}} \modelsBV{\text{B.energy} < 13} \ltlG{(\ltlAnd{\text{B.temp} \geq 30}{\text{B.temp} \leq 70})}
\end{flalign*}

It is easy to see that the formula we are verifying is also satisfied best-effort by Controller1 with upper-bound $13$.
Unfortunately, Controller1 is not optimal because there exists other controller which are satisfied with a lower upper-bound. 
One among them is Controller2 which satisfies the formula keeping the energy always lower than $6$.

\begin{lstlisting}[language=smv, caption=Best-effort semantics: Boiler example (3)]
MODULE Controller2(temp)
VAR
    on : boolean;
ASSIGN
    init(on) := FALSE;
    next(on) := case
        temp < 40 : TRUE;
        temp > 41 : FALSE;
        TRUE : on;
    esac;
\end{lstlisting}
